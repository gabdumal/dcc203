\section{Desenvolvimento}%
\label{sec:desenvolvimento}

\subsection{Ciência}

Compreender as bases da ciência é fundamental para a realização de trabalhos científicos.
Ela é definida por qualquer sistema de conhecimento baseado na observação e experimentação sistemática de fenômenos~\cite{britannica_2024_science}.

Este processo é descrito como o método científico.
Em sua aplicação trivial, o pesquisador desenvolve uma hipótese inspirada por observações, que deve ser construída de forma a permitir sua falseabilidade.
Sua estrutura é a de um tema geral, a determinação de um problema e a descrição de um resultado esperado~\cite{liamara_pesquisa}.

A partir desse processo, é possível testar a hipótese sistematicamente e obter resultados reproduzíveis.
Ao agregar o conhecimento científico, pode-se estabelecer novos modelos e teorias que explicam melhor os fenômenos inicialmente observados~\cite{britannica_2024_scientific_method}.

\subsection{Pesquisa científica}

Com base no método científico, pode-se conduzir um processo planejado de investigação, denominado pesquisa.
Ela descreve procedimentos de obtenção dos dados, formulação de hipóteses e testes, análise dos resultados e conclusões~\cite{erol_2017_conduct}.

Nesse sentido, verificam-se diferentes aplicações da pesquisa, sobre as quais ela pode ser classificada~\cite{liamara_pesquisa,gil_2002_projetos}.

\subsubsection{Natureza}

Quanto à natureza, realizam-se a pesquisa básica, que busca gerar conhecimentos fundamentais necessários para o avanço da ciência, e a pesquisa aplicada, que se debruça sobre a solução de problemas específicos~\cite{liamara_pesquisa}.

\subsubsection{Abordagem}

Quanto à abordagem, a pesquisa pode ser quantitativa ou qualitativa.

A pesquisa quantitativa destina-se a descrever uma situação a partir da medição numérica das hipóteses levantadas.
Passa-se então a uma análise estatística para gerar resultados exatos a partir dos dados coletados~\cite{liamara_pesquisa}.

Já a pesquisa qualitativa estuda descrições, comparações e interpretações de objetos não mensuráveis, de forma participativa.
O resultado é uma visão holística do fenômeno, sistematizando as informações coletadas~\cite{liamara_pesquisa}.

\subsubsection{Objetivos}

Quanto aos objetivos, a pesquisa pode ser exploratória, descritiva ou explicativa.

A pesquisa exploratória visa a proporcionar maior familiaridade com o problema.
Envolve levantamento bibliográfico, entrevistas e análise de exemplos.
Trabalhos dessa natureza incluem pesquisas bibliográficas e estudos de caso~\cite{liamara_pesquisa}.

Por sua vez, a pesquisa descritiva tem seu foco nas características de determinado fenômeno, sobre as quais se estabelecem variáveis.
É possível então identificar relações entre elas.
Para tal, é necessário aplicar técnicas padronizadas de coleta de dados, como em levantamentos~\cite{liamara_pesquisa}.

Por fim, a pesquisa explicativa busca identificar fatores que determinam ou contribuem para a ocorrência de fenômenos.
Ela está preocupada com explicações mais aprofundadas, requerendo os métodos experimental ou observacional.
Exemplos incluem pesquisas experimentais e \textit{ex-post-facto}~\cite{liamara_pesquisa}.

\subsubsection{Procedimentos técnicos}

Quanto aos procedimentos técnicos, a pesquisa pode ser bibliográfica, de campo, documental ou de laboratório.

A pesquisa bibliográfica visa a localizar e consultar fontes de informações diversas acerca de um tema específico.
Seu objetivo é o levantamento de conhecimento já produzido sobre o assunto~\cite{liamara_pesquisa}.

Por outro lado, a pesquisa de campo se baseia na coleta dados no ambiente e na situação em que ocorrem.
Para tal, deve-se definir o objeto de apreensão e a metodologia da investigação, realizando a coleta de forma sistemática, então procedendo ao registro, seleção de dados e organização.
Recursos que a caracterizam incluem a observação direta extensiva --- como a aplicação de questionários, que tem caráter quantitativo, e de formulários, que permitem análise amostral ---, além da intensiva --- caracterizada por entrevistas, para coleta qualitativa, e por observações, em que o pesquisador analisa a realidade investigada explorando os recursos do sentido~\cite{liamara_pesquisa}.

Outro tipo de pesquisa é a documental, que busca elaborar interpretações novas ou complementar as já existentes a partir da investigação de materiais que não receberam tratamento analítico.
Esses podem ser documentos, relatórios, reportagens ou produtos de mídia~\cite{liamara_pesquisa}.

Finalmente, a pesquisa de laboratório explora diretamente fenômenos em condições controladas a partir da experimentação.
Dessa forma, o pesquisador sistematicamente seleciona variáveis e as manipula, observando os resultados obtidos~\cite{liamara_pesquisa}.


\subsection{Trabalhos científicos}