
\documentclass[12pt]{article}

% -----
% Pacotes fundamentais 
% -----
\usepackage[portuguese]{babel}	% Idioma do documento
\usepackage[utf8]{inputenc}		% Codificação do documento (conversão automática dos acentos)
\usepackage[T1]{fontenc}		% Seleção de códigos de fonte.
\usepackage{hyphenat}
\usepackage{cite}
\usepackage{amsmath,amssymb,amsfonts}
\usepackage{algorithmic}
\usepackage{graphicx}
\usepackage{textcomp}
\usepackage{xcolor}
\usepackage[subentrycounter,seeautonumberlist,nonumberlist=true,acronym,nohypertypes={index}]{glossaries} % Glossário.
\usepackage[portuguese]{todonotes}
\usepackage{url}
% -----

% -----
% Configurações dos pacotes
% -----
\def\BibTeX{{\rm B\kern-.05em{\sc i\kern-.025em b}\kern-.08em
T\kern-.1667em\lower.7ex\hbox{E}\kern-.125emX}}

% --- Glossaries ---
\newglossary[ilg]{index}{ind}{idx}{\indexname}
\newcommand*{\newterm}[2]{
    \newglossaryentry{#1}
    {type=index,name={#2},description={\nopostdesc}}
}
\makeglossaries{} % Habilite este comando para permitir a impressão dos glossários
\loadglsentries{glossaries}
\renewcommand*{\glsclearpage}{} % Evita quebra de página entre os glossários

% --- Todonotes ---
\setlength{\marginparwidth}{2cm}
\presetkeys{todonotes}{inline,backgroundcolor=yellow}{}
% -----
% Glossário e Siglas
\newacronym[
    plural={SIs},
    longplural={Sistemas de Informação},
]{si}{SI}{Sistema de Informação}
\newacronym{sbc}{SBC}{Sociedade Brasileira de Computação}
\newacronym[
    plural={TICs},
    longplural={Tecnologias da Informação e Comunicação},
]{tic}{TIC}{Tecnologia da Informação e Comunicação}
\newacronym[
    plural={DCNs},
    longplural={Diretrizes Curriculares Nacionais},
]{dcn}{DCN}{Diretriz Curricular Nacional}
\newacronym{mec}{MEC}{Ministério da Educação}
\newacronym{es}{ES}{Engenharia de Software}
     
\sloppy

\title{Áreas de atuação segundo os Referencias Teóricos do curso de Sistemas de Informação}

\author{Celso Gabriel Malosto\inst{1}}
\address{Departamento de Ciência da Computação, Universidade Federal de Juiz de Fora
  (UFJF)\\
  Rua José Lourenço Kelmer -- São Pedro -- Juiz de Fora -- MG -- Brasil
  \email{gabriel.malosto@estudante.ufjf.br}
}

\bibliography{sbc-template}

\begin{document}

\maketitle

\begin{abstract}

\end{abstract}

\begin{resumo}

\end{resumo}

\section{Introdução}%
\label{sec:introducao}

\section{Referenciais Teóricos}%
\label{sec:referenciais_teoricos}

A 


\begin{quotation}
    Um Sistema de Informação pode ser definido como um conjunto de componentes
    inter-relacionados que trabalham juntos para coletar (ou recuperar), processar, armazenar e distribuir
    informação para suporte à tomada de decisão, coordenação, controle, análise de problemas, visualização de
    situações complexas e criação de novos produtos em uma organização \apud{laudon2021management}{zorzo2017referenciais}.
\end{quotation}



\section{Engenharia de Software}%
\label{sec:engenharia_de_software}

\section{Modelagem de Dados}%
\label{sec:modelagem_de_dados}

\section{Pesquisa Operacional}%
\label{sec:pesquisa_operacional}

\section{Conclusão}%
\label{sec:conclusao}

\printbibliography{}

\end{document}
