
\documentclass[12pt]{article}

% -----
% Pacotes fundamentais 
% -----
\usepackage[portuguese]{babel}	% Idioma do documento
\usepackage[utf8]{inputenc}		% Codificação do documento (conversão automática dos acentos)
\usepackage[T1]{fontenc}		% Seleção de códigos de fonte.
\usepackage{hyphenat}
\usepackage{cite}
\usepackage{amsmath,amssymb,amsfonts}
\usepackage{algorithmic}
\usepackage{graphicx}
\usepackage{textcomp}
\usepackage{xcolor}
\usepackage[subentrycounter,seeautonumberlist,nonumberlist=true,acronym,nohypertypes={index}]{glossaries} % Glossário.
\usepackage[portuguese]{todonotes}
\usepackage{url}
% -----

% -----
% Configurações dos pacotes
% -----
\def\BibTeX{{\rm B\kern-.05em{\sc i\kern-.025em b}\kern-.08em
T\kern-.1667em\lower.7ex\hbox{E}\kern-.125emX}}

% --- Glossaries ---
\newglossary[ilg]{index}{ind}{idx}{\indexname}
\newcommand*{\newterm}[2]{
    \newglossaryentry{#1}
    {type=index,name={#2},description={\nopostdesc}}
}
\makeglossaries{} % Habilite este comando para permitir a impressão dos glossários
\loadglsentries{glossaries}
\renewcommand*{\glsclearpage}{} % Evita quebra de página entre os glossários

% --- Todonotes ---
\setlength{\marginparwidth}{2cm}
\presetkeys{todonotes}{inline,backgroundcolor=yellow}{}
% -----
% Glossário e Siglas
\newacronym[
    plural={SIs},
    longplural={Sistemas de Informação},
]{si}{SI}{Sistema de Informação}
\newacronym{sbc}{SBC}{Sociedade Brasileira de Computação}
\newacronym[
    plural={TICs},
    longplural={Tecnologias da Informação e Comunicação},
]{tic}{TIC}{Tecnologia da Informação e Comunicação}
\newacronym[
    plural={DCNs},
    longplural={Diretrizes Curriculares Nacionais},
]{dcn}{DCN}{Diretriz Curricular Nacional}
\newacronym{mec}{MEC}{Ministério da Educação}
\newacronym{es}{ES}{Engenharia de Software}
     
\sloppy

\title{Engenharia de Software como área de atuação segundo os Referencias Teóricos do curso de Sistemas de Informação}

\author{Celso Gabriel Malosto\inst{1}}
\address{Departamento de Ciência da Computação, Universidade Federal de Juiz de Fora
  (UFJF)\\
  Rua José Lourenço Kelmer -- São Pedro -- Juiz de Fora -- MG -- Brasil
  \email{gabriel.malosto@estudante.ufjf.br}
}

\bibliography{sbc-template}

\begin{document}

\maketitle

\begin{abstract}

    The Brazilian Computer Society defines the theoretical references for the Bachelor's degree in Information Systems according to the National Curricular Guidelines established by the Ministry of Education.
    Among the skills expected of graduates, the area of Software Engineering stands out.
    A co-dependency is identified between the skills required and developed by this discipline and others of broad formation.

\end{abstract}

\begin{resumo}

    A \acrshort{sbc} define os referências teóricos para os cursos de \acrlong{bsi} segundo as \acrshort{dcn}s estabelecidas pelo \acrshort{mec}.
    Entre as habilidades esperadas dos profissionais formados, destaca-se a área de \acrlong{es}.
    Identifica-se uma co-dependência entre as habilidades necessárias e desenvolvidas por essa disciplina a outras de formação ampla.

\end{resumo}

\section{Introdução}%
\label{sec:introducao}

O curso de \gls{bsi} é uma graduação reconhecida pela \gls{sbc}.
Ele tem como objetivo formar profissionais capazes de compreender e atuar nas organizações, utilizando \glspl{tic} para melhorar a qualidade de vida das pessoas.

O \gls{mec} estabelece as \glspl{dcn} para os cursos de \gls{si}, as quais definem as habilidades e competências esperadas dos profissionais formados.
Entre elas, identifica-se a área de \acrfull{es}.
Este trabalho apresenta os referenciais teóricos do curso de \gls{si} e como eles se relacionam com a \gls{es}.

\section{Referenciais Teóricos}%
\label{sec:referenciais_teoricos}

É importante reconhecer organizações, como descrito nas \glspl{dcn}, não somente como empresas, mas como um termo amplo, definido por dois aspectos.
Na perspectiva técnica, tratam-se de ``uma estrutura social formal e estável durante um período de tempo, que utiliza e processa recursos do ambiente para a produção de novos produtos''~\cite{zorzo2017referenciais}.
Já na perspectiva comportamental, são ``uma coleção de direitos, privilégios, obrigações e responsabilidades que são balanceadas por meio de resolução de conflitos''~\cite{zorzo2017referenciais}.

Nesse contexto é possível destacar alguns pontos importantes nas \glspl{dcn} estabelecidas pelo \textcite{mec2016dcn}, como a necessidade de formar profissionais com sólida formação nas áreas de Ciência da Computação, Matemática e Administração.
Além de serem capazes de compreender as implicações organizacionais e sociais de soluções tecnológicas.
Ainda, espera-se que eles sejam capazes de identificar oportunidades de inovação, estabelecer os requisitos necessários para propostas de solução e gerenciar projetos de desenvolvimento e implantação de \glspl{si}.

\section{Engenharia de Software}%
\label{sec:engenharia_de_software}

Percebe-se que as habilidades esperadas dos profissionais de \gls{si} relacionam-se diretamente com a área de \gls{es}, tratada como uma disciplina que lida com todos os aspectos da produção de software~\cite{sommerville2011software}.

Assim, elencam-se quatro atividades fundamentais, quais sejam: especificação, que define aquilo que o sistema deve fazer, seus requisitos e restrições; desenvolvimento, relacionado com o projeto e programação do sistema; validação, que verifica se o sistema está de acordo com as especificações; e evolução, que propões mudanças e melhorias conforme novas necessidades surgem~\cite{sommerville2011software}.

Destaca-se a importância da elicitação de requisitos como uma atividade crítica para o sucesso de um projeto.
Essa habilidade deve ser desenvolvida com profundo conhecimento de domínios diversos e de técnicas de identificação daqueles, como por entrevistas, questionários, observação e prototipação~\cite{sommerville2011software}.
Aliam-se ao processo os artefatos produzidos, como documentos e modelos, que são utilizados para comunicar ideias e informações entre os membros da equipe e com os clientes.

\section{Conclusão}%
\label{sec:conclusao}

Neste trabalho, apresentou-se a área de \gls{es} como uma disciplina fundamental para o curso de \gls{si}, conforme os referenciais teóricos estabelecidos pela \gls{sbc}, que seguem as \glspl{dcn} do \gls{mec}.
Concomitantemente, as demais habilidades desenvolvidas no curso capacitam o profissional para executar as atividades diversas relacionadas à \gls{es}.

A identificação dessa relação é importante para orientar a construção de \glspl{ppc} de \glspl{bsi}, a fim de garantir que o processo de formação viabilize a plena compreensão das disciplinas e o desenvolvimento das competências inter-relacionadas esperadas dos egressos.

\printbibliography{}

\end{document}
