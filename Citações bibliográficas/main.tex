\documentclass[12pt]{article}

\usepackage{sbc-template}
\usepackage{graphicx,url}
\usepackage[brazil]{babel}
\usepackage[utf8]{inputenc}  

     
\sloppy

\title{Áreas de atuação segundo os Referencias Teóricos do curso de Sistemas de Informação}

\author{Celso Gabriel Malosto\inst{1}}

\address{Departamento de Ciência da Computação -- Universidade Federal de Juiz de Fora
  (UFJF)\\
  Rua José Lourenço Kelmer -- São Pedro -- Juiz de Fora -- MG -- Brasil
  \email{gabriel.malosto@estudante.ufjf.br}
}

\begin{document}

\maketitle

\begin{abstract}

\end{abstract}

\begin{resumo}

\end{resumo}


\section{Introdução}%
\label{sec:introducao}

\section{Bacharelado em Sistemas de Informação}%
\label{sec:bacharelado}



\section{Engenharia de Software}%
\label{sec:engenharia_de_software}



\section{Modelagem de Dados}%
\label{sec:modelagem_de_dados}



\section{Pesquisa Operacional}%
\label{sub:pesquisa_operacional}



\section{Conclusão}%
\label{sec:conclusao}


% \begin{figure}[ht]
%     \centering
%     \includegraphics[width=.5\textwidth]{fig1.jpg}
%     \caption{A typical figure}
%     \label{fig:exampleFig1}
% \end{figure}

% \begin{figure}[ht]
%     \centering
%     \includegraphics[width=.3\textwidth]{fig2.jpg}
%     \caption{This figure is an example of a figure caption taking more than one
%         line and justified considering margins mentioned in Section~\ref{sec:figs}.}
%     \label{fig:exampleFig2}
% \end{figure}

% \begin{table}[ht]
%     \centering
%     \caption{Variables to be considered on the evaluation of interaction
%         techniques}
%     \label{tab:exTable1}
%     \includegraphics[width=.7\textwidth]{table.jpg}
% \end{table}


\bibliographystyle{sbc}
\bibliography{sbc-template}

\end{document}
