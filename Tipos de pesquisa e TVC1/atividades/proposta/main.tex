\documentclass[12pt]{article}

% -----
% Pacotes fundamentais 
% -----
\usepackage[portuguese]{babel}	% Idioma do documento
\usepackage[utf8]{inputenc}		% Codificação do documento (conversão automática dos acentos)
\usepackage[T1]{fontenc}		% Seleção de códigos de fonte.
\usepackage{hyphenat}
\usepackage{cite}
\usepackage{amsmath,amssymb,amsfonts}
\usepackage{algorithmic}
\usepackage{graphicx}
\usepackage{textcomp}
\usepackage{xcolor}
\usepackage[subentrycounter,seeautonumberlist,nonumberlist=true,acronym,nohypertypes={index}]{glossaries} % Glossário.
\usepackage[portuguese]{todonotes}
\usepackage{url}
% -----

% -----
% Configurações dos pacotes
% -----
\def\BibTeX{{\rm B\kern-.05em{\sc i\kern-.025em b}\kern-.08em
T\kern-.1667em\lower.7ex\hbox{E}\kern-.125emX}}

% --- Glossaries ---
\newglossary[ilg]{index}{ind}{idx}{\indexname}
\newcommand*{\newterm}[2]{
    \newglossaryentry{#1}
    {type=index,name={#2},description={\nopostdesc}}
}
\makeglossaries{} % Habilite este comando para permitir a impressão dos glossários
\loadglsentries{glossaries}
\renewcommand*{\glsclearpage}{} % Evita quebra de página entre os glossários

% --- Todonotes ---
\setlength{\marginparwidth}{2cm}
\presetkeys{todonotes}{inline,backgroundcolor=yellow}{}
% -----
% Glossário e Siglas
\newacronym[
    plural={SIs},
    longplural={Sistemas de Informação},
]{si}{SI}{Sistema de Informação}
\newacronym{sbc}{SBC}{Sociedade Brasileira de Computação}
\newacronym[
    plural={TICs},
    longplural={Tecnologias da Informação e Comunicação},
]{tic}{TIC}{Tecnologia da Informação e Comunicação}
\newacronym[
    plural={DCNs},
    longplural={Diretrizes Curriculares Nacionais},
]{dcn}{DCN}{Diretriz Curricular Nacional}
\newacronym{mec}{MEC}{Ministério da Educação}
\newacronym{es}{ES}{Engenharia de Software}
     
\sloppy

\title{Modelagem de ferramenta para geração de agentes inteligentes com uso em testes de jogos}

\author{%
    Celso Gabriel Malosto\inst{1}%
}
\address{Departamento de Ciência da Computação, Universidade Federal de Juiz de Fora
  (UFJF)\\
  Rua José Lourenço Kelmer -- São Pedro -- Juiz de Fora -- MG -- Brasil
  \email{gabriel.malosto@estudante.ufjf.br}
}

\bibliography{sbc-template}

\begin{document}

\maketitle

\begin{abstract}

\end{abstract}

\begin{resumo}

\end{resumo}

\section{Introdução}%
\label{sec:introducao}

A criação de jogos é iterativa e, de forma cíclica, faz uso de protótipos e testes para explorar sistemas e corrigir desequilíbrios~\cite{marcelo2009design, fullerton2019game}.
Esse processo de balanceamento é complexo e conta com estudos de equilíbrio matemático, progressão e imparcialidade~\cite{romeroGameBalance2021}.

Ao procedimento de coleta de dados de partidas nomeia-se \textit{playtest}.
Eles demandam muitos recursos humanos, destacando a dificuldade de manter focado um grupo de testadores devido ao desgaste e às tarefas repetitivas~\cite{trzewiczek2017}.

A fim de aprimorar o processo de desenvolvimento de jogos, propomos uma ferramenta chamada \gls{apts}.
Ela permitirá representar jogos arbitrários e gerar agentes inteligentes capazes de realizar a fase de \textit{playtesting} de forma automatizada.

Este trabalho apresenta a fundamentação teórica para o desenvolvimento da ferramenta, a proposta de implementação, sua classificação em diferentes critérios, e a conclusões e limitações esperadas.

\section{Fundamentação teórica}%
\label{sec:fundamentacao_teorica}

A fim de representar os estados de um jogo e suas transições, estudamos o método \gls{mcts}.
Nesse método, cada estado guarda todas as informações mutáveis da partida em determinado turno.
Eles são guardados em um nó da árvore de busca~\cite{kocsis2006bandit, coulom2006efficient}.

Por sua vez, os movimentos tomados pelos jogadores levam a transições entre os nós.
Os níveis se sucedem ciclicamente representando os jogadores do turno.
Para um jogo de dois jogadores, um nível par representa o primeiro jogador, enquanto um nível ímpar representa o segundo~\cite{mcts-swiechowski}.

A busca é realizada em um ciclo de quatro etapas: seleção de um nó folha segundo um critério, expansão desse em uma nova jogada, simulação da partida a partir daquele ponto, e retro-propagação das métricas obtidas.
Assim, pode-se calcular métricas durante a simulação de uma partida, orientando a escolha de movimentos~\cite{mcts-swiechowski}.

Pretende-se estudar o método de \gls{resnet} para aprimorar o ciclo de busca do \gls{mcts}.
Essa classe de modelos de \gls{ia} é derivada das redes neurais convolucionais e aplicada primariamente em domínios de reconhecimento de imagens~\cite{he2015deep,zewen2022convolutional}.

\section{Proposta}%
\label{sec:proposta}

Pretende-se adaptar o método de \gls{resnet} para inserir como entrada um estado de um jogo arbitrário.
Para tal, deve-se transformar as informações daquele em um formato de canais retangulares, similarmente a como os canais de cores são representados em imagens.

Para um jogo de tabuleiro, como o jogo-da-velha, pode-se criar um canal que guarda as posições das peças jogadas pelo primeiro jogador, outro para as peças do segundo jogador, e um terceiro para as posições vazias.

Espera-se que a rede neural seja capaz de atribuir uma pontuação para cada movimento possível a partir de um estado analisado.
Além disso, ela deve também predizer uma métrica que indica a confiança de que aquele estado levará a uma vitória.

Esses dois dados são importantes para substituir a fase de simulação da \gls{mcts}.
Todas os movimentos válidos podem ser expandidos, em vez de apenas um.
A cada, guardar-se-á a pontuação determinadas.
Como valor de retro-propagação, pode-se utilizar a métrica de confiança.

O objetivo da modificação nos métodos é o treinamento de agentes inteligentes que possam jogar satisfatoriamente um jogo arbitrário a partir de aprendizado por reforço independente.
Estes, após treinados, servirão como testadores artificiais dos jogos inseridos na ferramenta \gls{apts}.

Seguindo esse mesmo objetivo, foi desenvolvida a ferramenta AlphaZero, pelo braço de pesquisa DeepMind, do Google.
Inicialmente, o modelo simulava agentes que jogavam Go.
Esses eram treinados por aprendizado supervisionado.
Uma evolução no modelo foi capaz de aprender o jogo por meio de aprendizado por reforço, iniciando por jogadas aleatórias~\cite{silverMasteringGameGo2016}.

Os pesquisadores da DeepMind generalizaram o modelo para que ele aprendesse qualquer jogo de tabuleiro dadas as suas regras, ao que se nomeou AlphaZero.
Os pesos e vieses são ajustados por meio de resultados obtidos de partidas simuladas entre agentes~\cite{alphazero-deepmind, silverGeneralReinforcementLearning2018}.

Como fator de distinção do \gls{apts}, elenca-se os usuários finais, quais sejam os \textit{designers} de jogos.
Eles poderão utilizar linguagens de descrição de jogos para representar seus protótipos.
Então, a aplicação será capaz de gerar métricas sobre as partidas jogadas, como a taxa de vitórias para cada jogador, a quantidade de turnos média, entre outras.
Espera-se assim diminuir os custos e o tempo de desenvolvimento de jogos.

\subsection{Natureza}

O trabalho proposto se trata de uma pesquisa aplicada, apresentando diretamente como fim uma ferramenta de software.
O problema específico é a automatização da fase de \textit{playtesting}.
Para solucioná-lo, são estudados primariamente dois métodos da literatura, o \gls{mcts} e o \gls{resnet}.
A expectativa de ganho para área é a diminuição de custos e de tempo.

Espera-se que os autores dos jogos possam acessar o sistema pelos navegadores de internet.
Assim, não haverá necessidade de instalação de software adicional, nem de uso de ferramentas de linha de comando.

\subsection{Finalidade}

Esta pesquisa se classifica como exploratória, dado que visa a compreender os métodos de representação de jogos, de busca e de aprendizado de máquina por meio de sua experimentação no desenvolvimento da ferramenta \gls{apts}.

Temos como hipótese que é possível representar jogos arbitrários ainda na fase de prototipação de forma que agentes inteligentes aprendam a jogá-los de forma satisfatória.
Acreditamos também que a ferramenta proposta será capaz de auxiliar os autores a realizar testes automatizados e melhorar o processo de criação de jogos.

Para comprovarmos a hipótese de que os agentes serão satisfatórios, pretendemos realizar experimentos e extrair métricas de jogos de informação completa e bem compreendidos, como o jogo-da-velha e o xadrez.
Compararemos os resultados obtidos com aqueles extraídos da literatura, tanto sobre jogadores humanos, como de outros sistemas que utilizam agentes inteligentes.

\subsection{Abordagem}

\section{Conclusão}%
\label{sec:conclusao}


\printbibliography{}

\end{document}
