\section{Monografia}%
\label{sec:monografia}

A pesquisa científica leva à produção de trabalhos em formas específicas, que se adéquam à metodologia empregada.
O trabalho desenvolvido pela conclusão de um doutorado se denomina Tese, ao passo em que aquele de mestrado é a Dissertação.
A Monografia, por sua vez, é desenvolvida pelo pesquisador vinculado a um curso de graduação ou pós-graduação lato sensu.
Seu objetivo é descrever um fenômeno baseando-se em uma compilação teórica diversa sobre um único tema de investigação~\cite{dias_2009_como_escrever}.

\subsection{Planejamento}

A escrita deste tipo de trabalho se dá de forma contínua, a partir de um planejamento prévio~\cite{maillard_2010_escrever_monografias}.
Pode-se determinar o tema central pela elaboração inicial do título.
Então, determina-se a o objetivo da pesquisa a partir de uma pergunta central.
Pode-se então avaliar a relevância do projeto~\cite{dias_2009_como_escrever}.

Com o objetivo definido, dá-se início à estruturação do trabalho.
Pode-se lançar um sumário como hipótese de aprofundamento do tema, o qual será refinado ao longo da pesquisa.
Quanto mais específicos forem os subcapítulos, maior será a clareza do trabalho~\cite{maillard_2010_escrever_monografias}.
A partir dele, descreve-se as expectativas dos resultados na versão inicial da introdução~\cite{dias_2009_como_escrever}.

\subsection{Estrutura}

A estruturação se baseia no ciclo da pesquisa.
Tem-se início com a observação de um fenômeno que inspira o pesquisador a formular uma pergunta, cuja resposta será o trabalho.

A fim de respondê-la, o pesquisador deve se basear num referencial teórico, realizando uma revisão bibliográfica acerca do tema.
O resultado dessa etapa será a produção de hipóteses.

Para testá-las, é necessário definir a metodologia empregada.
Este processo depende da disponibilidade de dados e do nível de conhecimento prévio.
Um fator chave é decidir se optar-se-á por uma abordagem quantitativa ou qualitativa.

Segue-se, então, para a fase de coleta de dados, sejam primários (coletados diretamente) ou secundários (obtidos de materiais já existentes).

É necessário então classificá-los e analisá-los de forma sistemática, a fim de comprovar ou refutar as hipóteses iniciais.

Finalmente, pode-se concluir o trabalho, descrevendo os resultados obtidos e comparando-os com a literatura existente.
É necessário ainda descrever as limitações do trabalho e sugerir trabalhos futuros~\cite{dias_2009_como_escrever}.