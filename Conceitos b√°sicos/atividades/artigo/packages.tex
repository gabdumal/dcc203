% -----
% Pacotes fundamentais 
% -----
\usepackage[portuguese]{babel}	% Idioma do documento
\usepackage[utf8]{inputenc}		% Codificação do documento (conversão automática dos acentos)
\usepackage[T1]{fontenc}		% Seleção de códigos de fonte.
\usepackage{hyphenat}
\usepackage{cite}
\usepackage{amsmath,amssymb,amsfonts}
\usepackage{algorithmic}
\usepackage{graphicx}
\usepackage{textcomp}
\usepackage{xcolor}
\usepackage[subentrycounter,seeautonumberlist,nonumberlist=true,acronym,nohypertypes={index}]{glossaries} % Glossário.
\usepackage[portuguese]{todonotes}
\usepackage{url}
% -----

% -----
% Configurações dos pacotes
% -----
\def\BibTeX{{\rm B\kern-.05em{\sc i\kern-.025em b}\kern-.08em
T\kern-.1667em\lower.7ex\hbox{E}\kern-.125emX}}

% --- Glossaries ---
\newglossary[ilg]{index}{ind}{idx}{\indexname}
\newcommand*{\newterm}[2]{
    \newglossaryentry{#1}
    {type=index,name={#2},description={\nopostdesc}}
}
\makeglossaries{} % Habilite este comando para permitir a impressão dos glossários
\loadglsentries{glossaries}
\renewcommand*{\glsclearpage}{} % Evita quebra de página entre os glossários

% --- Todonotes ---
\setlength{\marginparwidth}{2cm}
\presetkeys{todonotes}{inline,backgroundcolor=yellow}{}
% -----